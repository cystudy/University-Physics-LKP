\documentclass{ctexbook}
\usepackage{subfiles}
\usepackage{geometry}
\geometry{
  a4paper,
  total={170mm,257mm},
  left=20mm,
  top=20mm
}
\usepackage{hyperref}
\hypersetup{
    colorlinks=true,
    linkcolor=blue,
    filecolor=magenta,      
    urlcolor=cyan,
}
 
\urlstyle{same}
\usepackage{afterpage}
\newcommand \blankpage{
    \null
    \thispagestyle{empty}
    \addtocounter{page}{-1}
    \newpage
}
\usepackage{amsmath}

\title{仲英书院朋辈辅导 \\ 大学物理下(期末)\\ 知识点整理}
\author{主编:顾纯铭 \\ 主办:仲英书院 \\ 承办:仲英书院学业辅导中心}
\date{2016 年 12 月}

\begin{document}

\maketitle

\tableofcontents

\chapter{热力学基础}

\section{知识点}

\subsection{准静态过程}

\begin{enumerate}
    \item 把研究的宏观物体称为热力学系统,也称系统、工作物质;而把与热力学系统相互作用的环境称为外界。
    \item 准静态过程:从一个平衡态到另一平衡态所经过的每一中间状态均可近似当作平衡态的过程。准静态过程在平衡态 p–V 图上可用一条曲线来表示
    \item 准静态过程功的计算 $ W = \int_{V_1}^{V_2}{pdV} $,气体所作的功等于 p–V 图上过程曲线下的面积,系统所作的功不仅与系统的始末状态有关,而且与路径有关,故功是过程量。
    \item 热量:系统与外界之间由于温差而传递的能量,热量也是过程量。
\end{enumerate}

\subsection{热力学第一定律}

\begin{enumerate}
    \item 理想气体的内能:理想气体不考虑分子间的相互作用,其内能只是分子的无规则运动能量(包括分子内原子间的振动势能)的总和,是温度的单值函数。
    
    内能是状态量 
    
    $$ E = E(T) = \frac{1}{2} v i R T $$
    
    理想气体内能变化与 $ C_{V, m} $ 的关系
    
    $$ dE = v C_{V, m} d T $$
    
    \item  热力学第一定律:系统从外界吸收的热量,一部分使系统的内能增加,另一部分使系统对外界做功。
    
    $ Q = E_2 - E_1 + W $,对于无限小过程 $dQ = dE + dW$(注意:各物理量符号的规定)
    
\end{enumerate}

\subsection{四个重要过程(等容、等压、等温、绝热)}
    
重点内容,与课本中的给出图像结合进行理解。熟悉不同过程的 P-V 图曲线变化。并能做到辨别区分。
    
\subsection{循环}

\begin{enumerate}
    \item 循环:系统经过一系列状态变化后,又回到原来的状态的过程叫循环。循环可用 p—V 图上的一条闭合曲线表示。
    
    热机: 顺时针方向进行的循环。热机效率:
    
    $$ \eta = \frac{W}{Q_1} = 1 - \frac{Q_2}{Q_1} $$
    
    致冷机: 逆时针方向进行的循环。制冷系数:

    $$ \eta = \frac{Q_2}{|W|} = \frac{Q_2}{Q_1 - Q_2} $$
    
    \item 卡诺循环: 系统只和两个恒温热源进行热交换的准静态循环过程.卡诺循环由两个等温过程和两个绝热过程组成。
    
    卡诺热机效率:
    
    $$ \eta = 1 - \frac{T_2}{T_1} $$
    
    卡诺致冷机致冷系数:
    
    $$ \eta = \frac{T_2}{T_1 - T_2} $$

\end{enumerate}

\subsection{热力学第二定律}

\begin{enumerate}
    \item 热力学第二定律的两种表达式:
    
    开尔文表述:不可能制造出这样一种循环工作的热机,它只使单一热源冷却来做   功,而不放出热量给其他物体,或者说不使外界发生任何变化。    
    
    克劳修斯表述:不可能把热量从低温物体自动传到高温物体而不引起外界的变化。
    
    热力学第二定律的实质:自然界一切与热现象有关的实际宏观过程都是不可逆的。

    \item 可逆与不可逆过程:
    在系统状态变化过程中,如果逆过程能重复正过程的每一状态,而不引起其他变  化,这样的过程叫做可逆过程。反之,称为不可逆过程。

    \item 卡诺定理:
    
    \begin{enumerate}
        \item 在相同高温热源和低温热源之间工作的任意工作物质的可逆机都具有相同的效率。
        
        \item 工作在相同的高温热源和低温热源之间的一切不可逆机的效率都不可能大于可逆机的效率。
    \end{enumerate}
\end{enumerate}

\subsection{熵\ 熵增原理}

\begin{enumerate}
    \item 熵:在可逆过程中,系统从状态 A 改变到状态 B,其热温比的积分是一态函数熵的增量。
    
    $ S_B - S_A = \int_A^B{\frac{dQ}{T}}$ 或 $ ds = \frac{dQ}{T} $
    
    \item 熵增原理:孤立系统的熵永不减少。孤立系统中的可逆过程,其熵不变;孤立系统中的不可逆过程,其熵要增加。

\end{enumerate}

\section{重点}

\begin{enumerate}
    \item 准静态过程功的计算;
    \item 热力学第一定律以及式中各物理量的符号规定;
    \item 四个(等体、等压、等温、绝热)过程的过程特点、过程方程、过程曲线、内能增量、所作的功以及热量变化。
    \item 卡诺循环原理和几种效率公式;
    \item 热力学第二定律的两种表达、卡诺定理和熵增加原理的条件和内容。
\end{enumerate}

\section{难点}

\begin{enumerate}
    \item 热力学第一定律以及式中各物理量的符号规定;
    \item 四个(等体、等压、等温、绝热)过程的过程特点、过程方程、过程曲线、内能增量、所作的功以及热量变化。
    \item 卡诺循环原理和几种效率公式。
\end{enumerate}

\subfile{about.tex}

\end{document}
